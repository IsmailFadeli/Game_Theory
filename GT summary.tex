\documentclass[a4paper,12pt]{article}
\usepackage{listings}
\begin{document}
\title{Game theory Summary}
\author{Ismail Fadelli}
\date{\today}
\maketitle

\section{Game Theory}
Game Theroy is the study of rational behaviour in situations involving interdependece as it may involve:
\begin{itemize}
\item Common interest (coordination);
\item Competing interests (rivalry);
\item Rational behaviour: players can do the best they can, in their own eyes;
\item Because of the players' interdependence, a rational decision in a game must be based on a prediction of others' responses;
\end{itemize}
\subsection{The parts of a Game} 
A Game consists of three parts : 
Players, Actions and Payoffs
\subsubsection{Players}
Players are the decision makers and they can be : People, Governments or Companies.
\subsubsection{Actions}
What can the players do ?
Decide when to sell a stock, decide how to vote or enter a bid in an auction...
\
\subsubsection{Payoffs}
Payoffs can represent the motivation of the players, for example : Do they care about profit ? or Do they care about other players ? 
\subsection{Defining Games} Games can be represented in two methods : Normal forms and Extensive Forms.
\subsubsection{Normal Form : matrix form or strategic form}
A normal form represents a list of what players get on function of their actions.
\subsubsection{Extensive Form}
An extensive forms includes timing of moves. 
Players move sequentially, represented as a tree.
\begin{itemize}
\item Chess: white player moves, then black player can see white's move and react...
\end{itemize}
Keeps track of what each player knows when he or she makes a decision :
\begin{itemize}
\item Poker: bet seqentially - what can a given player see when they bet. 
\end{itemize}
\subsection{Normal Form}
Finit, n-person normal form game  ⟨$N, A, u$⟩:
\begin{itemize}
\item Players: $ N = {1, ... , n} $ is a finite set of $n$, indexed by $i$.
\item Actions set for player $i A_i$
\subitem $a = (a_1,...,a_n) \in A = A_1 * ... * A_n $ is an action profile.
\input amssym.tex
\item Utility function or Payoff function for player $i: u_i : A $  $\to$, ${\Bbb{R}}$
\subitem $u = (u_1,..., u_n)$, is a profile of utility functions.
\end{itemize}
\subsection{Best Response and Nash Equilibrium}
\subsubsection{Best Response}
\title{\textbf{Definition:} }
$a_i^* \in BR(a_{-i}) $ iff $  \forall a_i \in A_i, u_i(a_i^*,a_{-i}) \geq u_i(a_i, a_{-i})$  
\subsubsection{Nash Equilibrium}
\title{\textbf{Definition :} }
$a = <a_1,...,a_n>$ is a "\textbf{pure strategy}" if $\forall i, a_i \in BR(a_{-i})$
\subsection{Dominant strategies}
let $s_i$ and $s_i^`$ be two strategies for player $i$, and let $S_{-i}$ be the set of all possible strategy profiles for other players.
\bigbreak
\title{\textbf{Definitions:} }
\begin{itemize}
\item $s_i$ \textbf{strictly dominates} $s_i^`$ if $\forall s_{-i} \in S_{-i}, u_i(s_i, s_{-i}) \>> u_i(s_i^`, s_{-i})$
\item $s_i$ \textbf{very weakly dominates} $s_i^`$ if $\forall s_{-i} \in S_{-i}, u_i(s_i, s_{-i}) \geq u_i(s_i^`, s_{-i})$
\item A strategy is called \textbf{dominant} if it dominates all others.
\item A strategy profile consisting of dominant strategies for every player must be a Nash Equilibrium.
\end{itemize}
\subsection{Pareto Optimality}
\title{\textbf{Definition:}}
An outcome $o^*$ is \textbf{Pareto-optimal} if there is no other outcome that Pareto-dominates it.
\section{Mixed Strategies and Nash Equilibrium}
\title{\textbf{Definition:}}
A strategy $s_i$ for agent $i$ as any probabiltiy distribution over the actions $A_i$.
\begin{itemize}
\item \textbf{pure strategy:} only one action is played with positive probability
\item \textbf{mixed strategy:} more than one action is played with positive probability
\bigbreak
these actions are called the support of the mixed strategy.
\item Let the set of all strategies for $i$ be $S_i$
\item let the set of all strategy profiles be $S = S_1 \times... \times S_n$
\end{itemize}
\subsection{Utility in Mixed Strategies}
In order to find the payoff if all the players follow mixed strategy profile $s \in S$ we can use the \textbf{expected utility} from decision theory: 
\begin{equation} u_i(s) = \sum_{a \in A}u_i(a)P(a|s)\end{equation}
\begin{equation} P(a|s) = \prod_{j \in N}s_j(a_j)\end{equation}
\subsection{Best Response and Nash Equilibrium}The definitions of best response and Nash equilibrium are generalized from actions to strategies. 

\title{\textbf{Definition (Best Response):} }
$$ s_i^* \in BR(s_{-i}) \textbf{ if }   \forall s_i \in S_i, u_i(s_i^*,s_{-i}) \geq u_i(s_i, s_{-i})$$ 

\title{\textbf{Definition (Nash Equilibrium):} }
$$s = <s_1,...,s_n>\textbf{ is  a }\textbf{Nash Equilibrium if } \forall i, s_i \in BR(s_{-i})$$

\title{\textbf{Theorem (Nash, 1950)}: } Every finite game has a Nash equilibrium.
\subsection{Computing Nash Equilibrium}
\paragraph{Two algorithms for finding NE }
\begin{itemize}
\item LCP(Linear Complementarity) [Lemke-Howson].
\item Support Enumeration Method [Porter et al].
\end{itemize}
\subsection{Complexity Analysis}
\title{\textbf{Theorem:}}
Computing a Nash Equilibrium is a \textbf{PPAD-complete}, this theorem has been proven for:
\begin{itemize}
\item for games $\geq$ 4 players;
\item for games with 3 players;
\item for games with 2 players;
\end{itemize}
\subsection{Summary of mixed strategies}
\begin{itemize}
\item Some games have mixed strategy Nash Equilibria.
\item A player must be indifferent between the actions he or she randomizes over.
\item Randomization happen in business interactions, society, sports...
\end{itemize}

\section{Game Theory in Python}
Here is some Python code:
\lstset{language=Python}
\lstset{frame=lines}
\lstset{caption={Creating a game}}
\lstset{label={lst:code_direct}}
\lstset{basicstyle=\footnotesize}
\begin{lstlisting}
pip install nashpy
import nashpy as nash
import numpy as np
A =[[2,0], [1,3]]
B =[[3,0], [1,2]]
g = nash.Game(A, B)
g
Bi matrix game with payoff matrices:

Row player:
[[2 0]
 [1 3]]
 
Column player:
[[3 0]
 [1 2]]
\end{lstlisting}
\section{Strictly Dominated Strategies and Iterated Removal}
\begin{tabular}{|l|l|}

\hline
Apples & Green \\
\hline
Straws & Red \\
\hline
Oranges & Orange \\
\hline
\end{tabular}
\bigbreak
\title{\textbf{Strictly Dominated Strategies}
\section{Maxmin Strategies} 
A \textbf{maxmin strategy} for a player $i$ is a strategy that maximizes $i$'s worst case payoff, where all other players (whom we denote $-i$) happen to play the strategies which cause the greatest harm to $i$.
\bigbreak
The \textbf{maxmin value} of the game for player $i$ is that minimum payoff guaranteed by a maxmin strategy.
\paragraph{Definition (Maxmin)}{
The \textbf {maxmin strategy}  for player $i$ is $arg$ $max_{s_i}$ $min_{s_{-i}}$ $u_i(s_1, s_2)$, and \textbf{maxmin value} for player $i$ is $max_{s_i} min_{s_{-i}}u_i(s_1, s_2).$}
\paragraph{Minmax Theorem (Von Neuman, 1928)}{In any finite, two player, zero sum game, in any Nash Equilibrium each player receives a payoff that is equal to both his maxmin value and his minmax value.}
\subsection{Computing Minmax}
Minmax is solvable with LP (for two players)

\begin{equation}\textbf{subject to} \sum_{K \in A_2}U_1(a_1^j, a_2^k)\times s_2^k \leq U_1  \end{equation}
\section{Extensive Forms}{The extensive form is an alternative representation that makes the temporal structure explicit.}
\begin{itemize}
\item{Perfect information extensive form games.}
\item{Imperfect information extensive form games.}
\end{itemize}
\title {\textbf{Definition}} A finite perfect information game in extensive form is defined by the tuple ($N, A, H, Z,\chi ,\rho, \sigma, u $)
where:
\begin{itemize}
\item{Players: $N$ is a set of $n$ players.}
\item{Actions: $A$ is set of actions.}
\item{Choice nodes and labels for these nodes: }
\begin{itemize}
\item{Choice nodes: $H$ is a set of non-terminal choice nodes.}
\item{Action function: $\chi : H \to 2^A $ assigns to each choice a set of actions.}
\item{Player function: $\rho : H \to N$ assigns to each non-terminal node $h$ a player $i \in N$ who chooses an action at $h$.}
\end{itemize}
\item{Terminal nodes: $Z$ is a set of terminal nodes, disjoint from $H$.}
\item{Successor function: $\sigma : H \times A \to H \cup Z$ maps a choice node and an action to a new choice node or terminal node such that for all $h_1, h_2 \in H$ and $a_1, a_2 \in A$, if $\sigma(h_1, a_1) = \sigma(h_2, a_2)$ then $h_1  = h_2$  and $a_1 = a_2$} 
\end{itemize} 
\section{Transportation processes problem}
\end{document}
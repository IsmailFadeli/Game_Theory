


\paragraph{}The current approach to mode choice behavior
in the perspective of expected utility theory or random utility theory. However, travelers evaluate the alternative modes by individual experience and attitude which are not considered in the expected utility theory or random utility theory models. Therefore, many alternative theories have been proposed, for example, prospect theory, cumulative prospect theory and regret theory. Among them, cumulative prospect theory draws the most attention because it describes the bounded rational behaviors under various conditions.
\section{Travel Demand Management}
One of the most important socio-economic problems in recent decades has been the optimization of an urban transport system. Furthermore, This type of problem mainly occurs in developing countries, and the reason behind it is the increasing rate of car ownership.\footnote{Khovako, 2014,.}. Which urges cities to realize transport strategies combating this effect and also to decrease the negative impacts of transportation on the environment \footnote{World Bank, 2011}.

\section{Choice Decision Elements}
\paragraph {}The framework for the choice process is that the individual determines the available alternatives(modes), next, evaluates the attributes of each alternative, and then, uses a decision rule to select an alternative from among the available alternatives (Ben-Akiva and Lerman, 1985, Chapter 3).\\
Further in this section, we see that the elements of a choice process are : the decision maker, the alternatives, the attributes of alternatives and the decision rule.

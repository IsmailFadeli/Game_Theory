The goal of this project was to build a travel mode choice model capable of embedding game theory, the simulation model can be used in further research as a starting point. The model assumes that the travelers use mixed strategy when making choices, hence, the probability of a mode being chosen is dynamic. 
\paragraph{}
The simulations used in this work modeling a set of travelers in making choices of four modes : car, bus, taxi, and rail. This model is based on a split mode framework often used to describe the choice behavior of travelers.
\paragraph{}
Game theory has proven a usefulness in modeling the relationship between travel mode choices and their payoffs through Nash Equilibrium. A decrease in a mode's payoff can increase its proportion. The evolution part of this study is the change in Nash Equilibrium of travel mode choice.
\paragraph{}
Although people often use multimodal transport, which combines two or more modes in the same travel, which this model lacks. This mechanism is often used to avoid traffic or lack of coverage in some areas.
\paragraph{}Finally, it would be interesting to add more components besides the one that this travel mode choice model have, which can be done by further research. Studying the dynamics of travel mode choice behavior could extract new patterns that can be out of the scope of this model.
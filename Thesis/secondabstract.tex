\documentclass[11pt,a4paper]{report}
\usepackage{arabtex}
\usepackage[utf8]{inputenc}
\usepackage[LFE,LAE]{fontenc}
\usepackage[arabic]{babel}


\begin{document}
\begin{center}


\begin{otherlanguage}{arabic}
يعد اختيار وضع السفر عملية ديناميكية معروفة جيدًا تميز أنظمة النقل المعقدة. غالبًا ما لا تأخذ النماذج الحالية لاختيار وضع السفر ، بما في ذلك النماذج الاحتمالية ، في الاعتبار جميع العوامل التي يمكن أن تساهم في هذه العملية. الهدف الرئيسي من هذه الدراسة هو تطوير نموذج لعبة تطوري قائم على الوكيل باستخدام نظرية اللعبة لمحاكاة عملية اتخاذ القرار لاختيار وضع السفر. تهدف هذه الدراسة على وجه الخصوص إلى محاكاة وتحليل ديناميكيات هذه الظاهرة. تم تنفيذ النموذج في برنامج  \textLR{NetLogo} باستخدام عوامل اصطناعية. تشير نتائج المحاكاة التي تم الحصول عليها إلى أنماط واقعية. النموذج المقترح لديه القدرة على استخدامه كجزء من أغراض تخطيط النقل في المدينة. 
\end{otherlanguage}
\end{center}
       \end{document}
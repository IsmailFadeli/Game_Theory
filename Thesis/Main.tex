\documentclass[12pt]{report}
\usepackage[utf8]{inputenc}
\usepackage{graphicx}
\usepackage{listings}
\usepackage{threeparttable}
\usepackage{booktabs}
\usepackage{amsmath}
\usepackage{amsfonts}
\usepackage{longtable}
\usepackage{amssymb}
\graphicspath{ {./Plots/} }
\setcounter{tocdepth}{3}
\setcounter{secnumdepth}{3}

\usepackage[french,arabic,USenglish]{babel}
\usepackage{algorithm}
\usepackage{algorithmic}
\usepackage{fancyhdr}
\usepackage{qtree}
\usepackage{tikz-qtree}
\usepackage[a4paper, total={6in, 8in}]{geometry}
\pagestyle{fancy}
\usepackage[style=mla,babel=hyphen,backend=biber]{biblatex}
\usepackage {fancybox}
\newtheorem{mydef}{Definition}

\newenvironment{dedication}
  {\clearpage           % we want a new page
   \thispagestyle{empty}% no header and footer
   \vspace*{\stretch{1}}% some space at the top 
   \itshape             % the text is in italics
   \raggedleft          % flush to the right margin
  }
  {\par % end the paragraph
   \vspace{\stretch{3}} % space at bottom is three times that at the top
   \clearpage           % finish off the page
  }


\pagenumbering{roman}
\usepackage[nottoc]{tocbibind}
\def\blankpage{%
      \clearpage%
      \thispagestyle{empty}%
      \addtocounter{page}{-1}%
      \null%
      \clearpage}
\begin{document}
\begin{titlepage}

   \begin{center}
   \thisfancypage{%}
\setlength{\fboxsep}{10pt}\doublebox}{}
       \vspace*{1cm}
 	   \Huge
       \textbf{A Travel Mode Choice Model Using Game Theory}
		\LARGE
		
       \vspace{0.5cm}
        
            
       \vspace{1.5cm}

       \textit{Fadeli Ismail}
\vfill
\normalsize
Supervisor : KACI Abdellah\\

       \vfill
       
            	\normalsize
       A thesis presented for the degree of\\
       Masters in Logistics and Transport Engineering
            
       \vspace{0.8cm}
     
       
            \Large
       Department of Transport and Logistics Engineering\\
       National High School of Technology \\
       ENST\\
       Algeria\\
       2020
            
   \end{center}
   

\end{titlepage}
\blankpage
\thispagestyle{empty}
\pagenumbering{roman}
\begin{dedication}
Dedicated to my parents.

\end{dedication}



\clearpage
\begin{center}


\section*{Acknowledgment}
\end{center}
\addcontentsline{toc}{chapter}{Acknowledgment}
\paragraph{}
Foremost, I would like to express my sincere gratitude to my supervisor Mr. Kaci Abdellah for the continuous support of my master study and research, for his patience, motivation, and enthusiasm. His guidance helped me in all the time of research and writing of this work.
Besides my supervisor, I would like to thank the rest of my thesis committee for accepting to review this work.
\paragraph{}
My sincere thanks also goes to Mr. Abdellah Bradat, Mr. Arrouf Nasreddine, and Ms. Imene Merdaci for offering me the internship opportunity in their group.
\paragraph{}
Last but not the least, I would like to thank my family: my parents for supporting me through this work. 
\clearpage
\begin{center}


\section*{Abstract}


\addcontentsline{toc}{chapter}{Abstract}
Decision making in travel mode choice is a well-known dynamic process that characterize complex transportation systems. Existing models of travel mode choice, including the logit and probit ones, often do not consider all the factors that can contribute to this process. The main objective of this study is to develop an agent-based evolutionary game model with use of game theory to simulate the decision-making process of the travel mode choice. Particularly, this study aims to simulate and analyze the dynamics of this phenomena. The model was implemented in NetLogo software, using artificial agents. The obtained simulation results indicate realistic patterns. The proposed model has the potential to be used as part of a city's transportation planning purposes.
\\
\vspace{0.2cm}
   \begin{otherlanguage}{french}
   \textbf{Résumé}\\
   \vspace{0.2cm}
Le choix du mode de déplacement est un processus dynamique bien connu qui caractérise les systèmes de transport complexes. Les modèles existants de choix du mode de déplacement, y compris les modèles probabilistes et les modèles logit, ne tiennent souvent pas compte de tous les facteurs pouvant contribuer à ce processus. L'objectif principal de cette étude est de développer un modèle de jeu évolutif à base d'agents avec l'utilisation de la théorie des jeux pour simuler le processus de prise de décision du choix du mode de déplacement. En particulier, cette étude vise à simuler et analyser la dynamique de ce phénomène. Le modèle a été implémenté dans le logiciel NetLogo, à l'aide des agents artificiels. Les résultats de simulation obtenus indiquent des modèles réalistes. Le modèle proposé a le potentiel d'être utilisé dans le cadre de la planification des transports urbain.
\vspace{0.2cm}
\end{otherlanguage}
  \\
  
   \begin{otherlanguage}{arabic}
   \textbf{ملخص}\\
 \vspace{0.2cm}
يعد اختيار وسيلة السفر عملية ديناميكية معروفة تميز أنظمة النقل المعقدة. غالبًا ما لا تأخذ النماذج الحالية لاختيار وضع السفر ، بما في ذلك النماذج الاحتمالية ، في الاعتبار جميع العوامل التي يمكن أن تساهم في هذه العملية. الهدف الرئيسي من هذه الدراسة هو تطوير نموذج لعبة تطوري قائم على الوكيل باستخدام نظرية الالعاب لمحاكاة عملية اتخاذ القرار لاختيار وضع السفر. تهدف هذه الدراسة على وجه الخصوص إلى محاكاة وتحليل ديناميكيات هذه الظاهرة. تم تنفيذ النموذج في برنامج  \textLR{NetLogo} باستخدام عوامل اصطناعية. تشير نتائج المحاكاة التي تم الحصول عليها إلى أنماط واقعية. النموذج المقترح لديه القدرة على استخدامه كجزء من تخطيط النقل في المدينة. 
\end{otherlanguage}
\\
\vspace{0.2cm}
\textbf{Keywords}\\
\vspace{0.2cm}
Game theory, transport modes, evolution, mode choice, agent based model.

\end{center}

\tableofcontents
\listoffigures
\listoftables


\clearpage


\pagenumbering{arabic}
\chapter*{Introduction}
\setcounter{page}{1}
\addcontentsline{toc}{chapter}{Introduction}
The grow of private vehicle use causes congestion that eventually  increases travel time, this increase pushes governments to motivate people towards public transport. However, more usage of public transport should be followed by an improvement of public transport services.\\

With the intention of making a transport demand analysis, it is essential to understand the traveler's mode choice behavior, where a demand is the accumulation of individuals decisions. The most important element in modeling a transport system is the mode split model, which provides a mathematical framework of the choices a traveler can have of which mode of transport is more suitable.\\

The different objectives in travel mode choice leads to the urge of applying game theory for making decisions based on finding the equilibrium of the passenger choices. However, game theory is rarely used in the transport field with several obstacles appearing in the transport characteristics. But, it is possible to predict the behaviors of travelers? this question has been asked over time, but we still do not have clear answers. Despite the common knowledge that human actions are random and unpredictable, human mobility follows certain patterns.\\

The purpose of this study is to describe how travelers adjust their mode of transport choice behaviors using an evolutionary game model. In evolutionary game theory, a dynamic process is set to describe how players adjust their choices overtime as they learn from the game and also from other players. The present document is organized into three chapters. Chapter 1 describes the basic theories applied in this work, including traditional game theory concepts and evolutionary dynamics. In Chapter 2, key studies from the literature regarding travel choice behavior are briefly examined. The third chapter describes the model used in this study. Limitations of the proposed modeling method and further research directions are discussed at the end. 

\fancyhf{}
\rhead{\thesection}
\lhead{\leftmark}
\fancyfoot[LE,RO]{\thepage}
\renewcommand{\headrulewidth}{2pt}
\renewcommand{\footrulewidth}{1pt}

\chapter{Game Theory and Evolutionary Dynamics}
\section{Game Theory}
Game Theory is the study of rational behavior in situations involving interdependence as it may involve:
\begin{itemize}
\item Common interest (coordination);
\item Competing interests (rivalry);
\item Rational behavior: players can do the best they can, in their own eyes;
\item Because of the players' interdependence, a rational decision in a game must be based on a prediction of others' responses;
\end{itemize}
\subsection{The parts of a Game} 
A Game consists of three parts : 
Players, Actions and Payoffs
\subsubsection{Players}
Players are the decision makers and they can be : People, Governments or Companies.
\subsubsection{Actions}
What can the players do ?
Decide when to sell a stock, decide how to vote or enter a bid in an auction...
\subsubsection{Payoffs}
Payoffs can represent the motivation of the players, for example : Do they care about profit ? or Do they care about other players ? 
\subsection{Defining Games} Games can be represented in two methods : Normal forms and Extensive Forms.
\subsection{Extensive Form}
An extensive form game includes timing of moves. 
Players move sequentially, represented as a tree.
\begin{itemize}
\item Chess: white player moves, then black player can see white's move and react...
\end{itemize}
Keeps track of what each player knows when he or she makes a decision :
\begin{itemize}
\item Poker: bet sequentially - what can a given player see when they bet. 
\end{itemize}
\subsection{Normal Form}
A normal form represents a list of what players get on function of their actions.
Finite, n-person normal form game  ⟨$N, A, u$⟩:
\begin{itemize}
\item Players: $ N = {1, ... , n} $ is a finite set of $n$, indexed by $i$.
\item Actions set for player $i$ $A_i$
\subitem $a = (a_1,...,a_n) \in A = A_1 * ... * A_n $ is an action profile.
\item Utility function or Payoff function for player $i: u_i : A $  $\to$ ${\Bbb{R}}$
\subitem $u = (u_1,..., u_n)$, is a profile of utility functions.
\end{itemize}
\subsection{Best Response and Nash Equilibrium}\label{subsection}
\paragraph{Best Response :}
\begin{equation}\label{eq:1}
 a_i^* \in BR(a_{-i})  iff   \forall a_i \in A_i, u_i(a_i^*,a_{-i}) \geq u_i(a_i, a_{-i})
\end{equation}
 
\paragraph{Nash Equilibrium (Definition):}
$a = <a_1,...,a_n>$ is a "\textbf{pure strategy}" if $\forall i, a_i \in BR(a_{-i})$
\subsection{Dominant strategies}
let $s_i$ and $s_i^`$ be two strategies for player $i$, and let $S_{-i}$ be the set of all possible strategy profiles for other players.
\bigbreak
\title{\textbf{Definitions:} }
\begin{itemize}
\item $s_i$ \textbf{strictly dominates} $s_i^`$ if $\forall s_{-i} \in S_{-i}, u_i(s_i, s_{-i}) \>> u_i(s_i^`, s_{-i})$
\item $s_i$ \textbf{very weakly dominates} $s_i^`$ if $\forall s_{-i} \in S_{-i}, u_i(s_i, s_{-i}) \geq u_i(s_i^`, s_{-i})$
\item A strategy is called \textbf{dominant} if it dominates all others.
\item A strategy profile consisting of dominant strategies for every player must be a Nash Equilibrium.
\end{itemize}
\subsection{Pareto Optimal}
\title{\textbf{Definition:}}
An outcome $o^*$ is \textbf{Pareto-optimal} if there is no other outcome that Pareto-dominates it.
\section{Mixed Strategies and Nash Equilibrium}
\title{\textbf{Definition:}}
A strategy $s_i$ for agent $i$ as any probability distribution over the actions $A_i$.
\begin{itemize}
\item \textbf{pure strategy:} only one action is played with positive probability
\item \textbf{mixed strategy:} more than one action is played with positive probability
\bigbreak
these actions are called the support of the mixed strategy.
\item Let the set of all strategies for $i$ be $S_i$
\item let the set of all strategy profiles be $S = S_1 \times... \times S_n$
\end{itemize}
\subsection{Utility in Mixed Strategies}
In order to find the payoff if all the players follow mixed strategy profile $s \in S$ we can use the \textbf{expected utility} from decision theory: 
\begin{equation} u_i(s) = \sum_{a \in A}u_i(a)P(a|s)\end{equation}
\begin{equation} P(a|s) = \prod_{j \in N}s_j(a_j)\end{equation}
\subsection{Best Response and Nash Equilibrium}The definitions of best response and Nash equilibrium are generalized from actions to strategies. 

\paragraph{Definition (Best Response): }
\begin{equation}
s_i^* \in BR(s_{-i}) \textbf{ if }   \forall s_i \in S_i, u_i(s_i^*,s_{-i}) \geq u_i(s_i, s_{-i})
\end{equation}


\paragraph{Definition (Nash Equilibrium)}
$$s = <s_1,...,s_n>\textbf{ is  a }\textbf{Nash Equilibrium if } \forall i, s_i \in BR(s_{-i})$$

\paragraph{Theorem (Nash, 1950)} Every finite game has a Nash equilibrium.
\subsection{Computing Nash Equilibrium}
\paragraph{Two algorithms for finding NE }
\begin{itemize}
\item LCP(Linear Complimentary) [Lemke-Howson].
\item Support Enumeration Method [Porter et al].
\end{itemize}
\subsection{Complexity Analysis}
\title{\textbf{Theorem:}}
Computing a Nash Equilibrium is a \textbf{PPAD-complete}\footnote{PPAD : Polynomial Parity Argument on Directed Graphs} , this theorem has been proven for:
\begin{itemize}
\item for games $\geq$ 4 players;
\item for games with 3 players;
\item for games with 2 players;
\end{itemize}
\subsection{Summary of mixed strategies}
\begin{itemize}
\item Some games have mixed strategy Nash Equilibria.
\item A player must be indifferent between the actions he or she randomizes over.
\item Randomization happen in business interactions, society, sports...
\end{itemize}

\subsection{Strictly Dominated Strategies}
\paragraph{Definition}a strategy $a_i \in A_i $ is strictly dominated by $a'_i \in A_i$ if
\begin{equation} u_i(a_i, a_{-i}) < u_i(a'_i, a_{-i}) ,  \forall   a_{-i} \in A_{-i} \end{equation}
\subsection{Weakly Dominated Strategies}
\paragraph{Definition}a strategy $a_i \in A_i $ is weakly dominated by $a'_i \in A_i$ if
\begin{equation} u_i(a_i, a_{-i}) \leq u_i(a'_i, a_{-i}) ,  \forall   a_{-i} \in A_{-i} \end{equation}
\begin{center}
and 
\end{center}

\begin{equation} u_i(a_i, a_{-i}) < u_i(a'_i, a_{-i}) ,\exists   a_{-i} \in A_{-i} \end{equation} 
\section{Perfect inf}{The extensive form is an alternative representation that makes the temporal structure explicit.}
\begin{itemize}
\item{Perfect information extensive form games.}
\item{Imperfect information extensive form games.}
\end{itemize}
\title {\textbf{Definition}} A finite perfect information game in extensive form is defined by the tuple ($N, A, H, Z,\chi ,\rho, \sigma, u $)
where:
\begin{itemize}
\item{Players: $N$ is a set of $n$ players.}
\item{Actions: $A$ is set of actions.}
\item{Choice nodes and labels for these nodes: }
\begin{itemize}
\item{Choice nodes: $H$ is a set of non-terminal choice nodes.}
\item{Action function: $\chi : H \to 2^A $ assigns to each choice a set of actions.}
\item{Player function: $\rho : H \to N$ assigns to each non-terminal node $h$ a player $i \in N$ who chooses an action at $h$.}
\end{itemize}
\item{Terminal nodes: $Z$ is a set of terminal nodes, disjoint from $H$.}
\item{Successor function: $\sigma : H \times A \to H \cup Z$ maps a choice node and an action to a new choice node or terminal node such that for all $h_1, h_2 \in H$ and $a_1, a_2 \in A$, if $\sigma(h_1, a_1) = \sigma(h_2, a_2)$ then $h_1  = h_2$  and $a_1 = a_2$} 
\item{Utility function: $u = (u_1,...,u_n)$ where $u_i : Z \to R$
}
\end{itemize} 
\paragraph{}figure \ref{fig:scaled_diss} shows a sharing game represented in the extensive form
\begin{figure}[h]
 
  \centering
  \begin{tikzpicture}[baseline] % baseline makes the example number stay at the top of the tree
   \Tree[.1 [.\textit{2 } [.\textit{(0,0) } ] [.\textit{(2,0) } ]][.\textit{2 } [.\textit{(0,0) } ] [.\textit{(1,1) } ]][.\textit{2 } [.\textit{(0,0) } ] [.\textit{(0,2) } ]]]
     \end{tikzpicture}%
  \caption{Sharing Game\label{fig:scaled_diss}}
\end{figure}
\subsection{Pure Strategies}
\paragraph{} A pure strategy for a player in a perfect-information game is a complete specification of which action to take at each node belonging to that player.
\paragraph{Definition} Let $G = (N, A, H, Z,\chi ,\rho, \sigma, u ) $ be a perfect-information extensive-form game. Then the pure strategies of player $i$ consist of the cross product\\
\begin{center}
$ \prod_{h \in H, \rho(h)=i}\chi(h)$
\end{center}
\paragraph{}
Given our new definition of pure strategy, we can reuse our old definitions of mixed strategies and Nash equilibrium in \ref{eq:1}.
\subsection{Sub-game Perfection}

\begin{mydef}[Sub-game Perfection]\label{def:def555}
The set of sub-games of $G$ is defined by the sub-games of $G$ rooted at each of the nodes in $G$.
\end{mydef}
\paragraph{}Let $s$ be a  sub-game perfect equilibrium of $G$ if for any sub-game $G'$ of $G$, the restriction of $s$ to G' is a Nash Equilibrium of $G'$. Since $G$ is its own sub-game , every sub-game perfect is a Nash Equilibrium.
\subsection{Backward Induction}
\paragraph{}Backward Induction has been used in solving games since John von Neumann and Oskar Morgenstern published their book, Theory of Games and Economic Behaviors in 1944.
\paragraph{}The idea behind Backward Induction is to identify the equilibrium in the buttom trees, and adopt these as one moves up the tree as the next algorithm shows.

\begin{algorithm}
\caption{Backward Induction\label{fig:scaled_back}}
\begin{algorithmic}
\RETURN $u(h)$
\IF{$h \in Z$}
\RETURN{$u(h)$}
\ENDIF
\STATE $best-util \leftarrow -50$
\FORALL{$a \in \rho(h)$} 
\STATE $util-at-child \leftarrow BACKWARDINDUCTION(\sigma(h,a))$ 
\IF{$util-at-child_p(h) >best-util_p(h)$}
\STATE $best-util \leftarrow util-at-child$
\ENDIF
\ENDFOR
\RETURN $best-util$
\end{algorithmic}
\end{algorithm}

\paragraph{} Denote $util_at_child$ is a utility vector for each player.
\section{Evolutionary Game Theory}

\paragraph{}Evolution and Game Theory was introduced by John Maynard Smith in Evolution and The Theory of Games. The Theory was formulated to understand the behavior of animals in game theoretic situations. But it can be applied to modeling human behavior.

\paragraph{}After the emergence of traditional game theory, biologists realized the potential of game theory to formally study adaptation and convolution of biological populations, especially in contexts where the fitness of a phenotype depends on the composition of the population (Hamilton, 1967). The main assumption of evolutionary game theory was that strategies with greater payoffs at a particular time would tend to spread more and thus have better chances of being present in the future.
\paragraph{}The most important concept of evolutionary thinking that was introduced by Manynard Smith and Price (1973) is the notion of \textbf{Evolutionary Stable Strategy}(ESS), for 2-player symmetric games played by individuals belonging to the same population. Furthermore, a strategy $s$ is an ESS if and only if, when adopted by all members of a population, meaning that any other strategy $i$ that could enter the population in a low percentage would obtain a strictly  lower expected payoff in the population than the $s$ strategy.
\paragraph{}The basic ideas behind Evolutionary game theory is that strategies with greater payoffs tend to spread more, and that fitness is frequency dependent soon transcended the borders of biology and started to spread through many other disciplines. In economic context, it was understood that natural selection would derive from competition among entities for small resources or market shares. In social contexts, evolution was often understood as cultural evolution, and it referred to dynamic changes in behavior or ideas over time (\cite{Nelson and Winter, 1982})(\cite{Boyd and Richerson, 1985}).
\paragraph{}In order to extend this understanding further, let's consider this example:
Suppose that a small group of mutants choosing a strategy different from $\delta$* to enter the population.
\begin{itemize}
\item Denote the fraction of mutants in the population by $\varepsilon$ and assume that the mutant adopts the strategy $\delta$.
\item The expected payoff of a mutant is : 
	$(1-\varepsilon)u(\delta,\delta*)+\varepsilon u(\delta*,\delta)$
\item The expected payoff of a mutant that adopts the strategy is :	\\
	$(1-\varepsilon)u(\delta*,\delta*)+\varepsilon u(\delta*,\delta)$
	\item For any mutation to be driven out of the population we need the expected payoff of any mutant to be less than the expected payoff of normal organism :\\
	\begin{equation}(1-\varepsilon)u(\delta*,\delta*)+\varepsilon u(\delta*,\delta) > (1-\varepsilon)u(\delta,\delta*)+\varepsilon u(\delta*,\delta)  \end{equation}
\end{itemize}



\chapter{Literature Survey and Methodology} 



\paragraph{}The current approach to mode choice behavior
in the perspective of expected utility theory or random utility theory. However, travelers evaluate the alternative modes by individual experience and attitude which are not considered in the expected utility theory or random utility theory models. Therefore, many alternative theories have been proposed, for example, prospect theory, cumulative prospect theory and regret theory. Among them, cumulative prospect theory draws the most attention because it describes the bounded rational behaviors under various conditions.

\clearpage
\section{Travel Choice Models}
\paragraph{}Many models are available for analyzing data of travel mode choice. However, three main models have been dominant: logit models, probit models, and discriminant models. These simple choice models are described first. Mode-use models are different from other mode choice models in their dependent variables and model structure. In the third part of this section, we discuss some studies that have used psychological scaling models to probe more deeply into the nature of mode choice process. This is followed by a discussion of reliability and validity analysis in mode choice models.

\subsection{Simple Choice Models}
\paragraph{}Three simple-choice models are usually discussed in the context of utility theory. According to this understanding, the utility $U_i$ of alternative mode $i$ is expressed as the sum of a deterministic component $V_i$ and a random component $\epsilon_i$(\cite{Richard, David, 1982}):
\begin{equation}
U_i = V_i + \epsilon_i
\end{equation}
\paragraph{} The probability of choosing the $i$th mode from a set of $n$ alternatives is thus:
\begin{equation}
P_i = P_r[U_i>U_j](j=i)
\end{equation}
Alternatively, 
\begin{equation}\label{eq:3}
P_i = P_r[\epsilon_j < V_i - V_j + \epsilon_i](j=i)
\end{equation}
\paragraph{}If the cumulative density function of the error $\epsilon =(\epsilon_1, . . . , \epsilon_n)$ is $F(t_1, . . . , t_n)$, and the partial of the cumulative density function with respect to variable $i$ is $F_i(t_1, . . . , t_n)$, then equation \ref{eq:3} becomes:
\begin{equation}\label{eq:4}
P_i = 	\int_{-\infty}^{+\infty} F_i(. . . , t + V_i - V_j, . . .)df
\end{equation}
\paragraph{}If the error terms are independent identically distributed Gumbel variate, then Equation \ref{eq:4} is a multivariate logit model. If the error terms have a joint multivariate normal distribution, then \ref{eq:4} defines a multinomial probit model. 
\paragraph{}The third simple-choice model, discriminant analysis, was originally developed for taxonomic purposes. However, discriminant analysus has been avoided in mode choice analysis because it lacks the probabilistic theory that is possessed by other behavioral-choice models. In recent decades, logit models have been the most used when it comes to travel mode choice analysis.
\paragraph{}It is also important to note that probit models are always associated with maximum-likelihood procedures, and discriminant models are always associated with least-squares procedures.

\subsection{Mode Use Models}
\paragraph{}These models seek to explain the degree of actual or anticipated use for a given mode. Models of this type do not fit into the travel-demand models of planners as well as mode choice models, but they are legitimate means of investigating the behavioral determinants and relations of mode selection.
\paragraph{}Mode use models vary in complexity from single equation models that explain the frequency of mode use or customer satisfaction with a particular mode, to more complex multi-equation models that investigate the structure of the mode choice process. An example would be the study by Dobson, Dunbarn Smith, Reibstein, and Lovelock (1978) that used structural equations on cross-sectional data to try to determine the casual relations between transportation attitudes and behavioral responses. Another study done by Tischer and Philips(1979), have used quasi-experimental designs employing time series data to measure the patterns of causality.

\subsection{Scaling Models}
\paragraph{}Although the psychological models of Juce (1959) and Thurstone(1927) are often used to justify the use of multinomial logit model, these individual choice models are rarely used to investigate the mode choices of individuals.  The reason for the absence of psychological models in transportation may be the modest results that were reported in early studies in which deterministic vector models were used to analyze subjects preferences.
\paragraph{}Mode use and scaling models have expanded our knowledge of the mode choice process. The simple-choice, mode-use, and scaling models utilize different types of data to explain mode choice at different levels of analysis. Models developed for psychological stimuli cannot just be taken off the shelf and applied to complex situations like mode choice without modification.

\subsection{Reliability and Validity}
\paragraph{}Reliability and validity testing is very important in mode choice models. Early applications of logit analysis were largely descriptive in nature. Later applications became more sophisticated in their use of statistical procedures.

\section{Travel Demand Management}
\paragraph{}One of the most important socio-economic problems in recent decades has been the optimization of an urban transport system. Furthermore, This type of problem mainly occurs in developing countries, and the reason behind it is the increasing rate of car ownership.\footnote{Khovako, 2014,.}. Which urges cities to realize transport strategies combating this effect and also to decrease the negative impacts of transportation on the environment \footnote{World Bank, 2011}.

\section{Choice Decision Elements}
The framework for the choice process is that the individual determines the available alternatives(modes), next, evaluates the attributes of each alternative, and then, uses a decision rule to select an alternative from among the available alternatives (Ben-Akiva and Lerman, 1985). Further in this section, we see that the elements of a choice process are : the individuals, travel modes, the attributes of modes and the decision rule.

\section{Mode Characteristics}
\paragraph{}The travel mode choice (TMC) is an important step of the transportation forecasting (\cite{Litman,2011}). The main modes for travelers are private cars or public transportation. TMC is usually mathematically represented by logit functions, due to its consideration of particular qualities of travelers(\cite{Bravo et al, 2009}).
\subsection{Time and Costs}
\paragraph{}Travel and costs are the two most commonly investigated determinants of travel-mode choice. Studies done by Lisco (1967) and Quarmby (1967) used travel time and travel cost differences as two independent variables in their models. Travel time and cost differences are discussed in the next section. Another method was used by Warner (1962), who used travel time and cost as ratios.
\paragraph{}Watson (1974) believed that the difference formulation is most appropriate for between city trips, but when intercity trips are being analyzed other factors may be in order. On longer intercity trips, it is difficult to say whether a traveler would base their mode choice on time, whereas the preference for faster modes is a reasonable assumption on a short commuting trip. 
\paragraph{}Many studies have made the specification of the time and cost variables between overall travel time and excess travel time. This distinction arose on the assumption that time spent in different ways while traveling may be valued differently. A study by Quarmby (1967) divided travel time into "travel time" and "excess travel time", mentioning that the excess out of vehicle time on a journey may be greater for bus than car users. An important assumption made by Ben-Akiva and Richards (1976), that in vehicle time is generally viewed the same for all modes, whereas out of vehicle time tends to be mode  specific. 
\paragraph{}Travel cost has been discussed in detail by Gillen (1977), who notes that many mode choice studies gave added the cost of parking to automobile running costs (Williams, 1978). Gillen found that parking cost is a  crucial variable if the study aims to obtain unbiased estimates of operating costs on mode choice. 
\paragraph{}It is still unknown of which costs are relevant to mode choice decision. The microeconomic theory that underlies the specification of these models suggests that "marginal operating costs" are the relevant costs. However, from the consumer's perspective, the total cost of ownership including purchase price and maintenance costs may be the more important consideration.  
\paragraph{}The next function computes the probability of car mode choice, which depends on the difference between the travel time $\sigma_t$ and the travel costs $\sigma_c$:
\begin{equation}
p^{*}(\sigma_t,\sigma_c) = \frac{1}{1+exp(a_0+ a_i c^c + a_c \sigma_c)}
\end{equation}
\paragraph{}The objective function based on logit function may is presented in the following equation(\cite{Hollander et al, 2006}):
\begin{equation}
G(p) = (p - p^{*}(\sigma_t,\sigma_c))*
	 = (p - \frac{1}{1+exp(a_0+ a_i c^c + a_c \sigma_c)})^2 \rightarrow min_p.
\end{equation}
\paragraph{}The difference of the travel times between modes (private
and public transport) for traveling between A and B :
\begin{equation}
\sigma t_{a,b} = T^{r}_a + T^{r}_b + T^{p}_b - T^{t}_a - T^{t}_b - t_w
\end{equation}
where $t_w$ is the waiting time in public transport.
\paragraph{}The difference between travel costs could be : 
\begin{equation}
\sigma c_{a,b} = c^{c} ( t^{r}_b + t^{r}_b - t^{p}_b )- c^{p}_b - c^t
\end{equation}

\section{Evolutionary Game Theory and Engineering}
\paragraph{}Many Engineering Infrastructures are becoming increasingly complex to manage due to large scale distributed nature and the nonlinear interdependence between their components (\cite{Quijano et al, 2017}).Including transportation systems, communication networks, data networks, and teams of anonymous vehicles. Controlling these large scale distributed systems requires the implementation of decision rules for interconnected components that grantee the accomplishment of a collective objective in an environment that is often dynamic and uncertain. In order to achieve this goal, traditional control theory is often of little use, since distributed systems generally lack a central  entity with access or control over all components(\cite{Marden and Shamma, 2015}).

\chapter{Model and Analysis}

The contribution of this study on travel mode choice is the research on modeling choice behaviors in travel mode selection using game theory concepts, and using an evolutionary analysis to determine the behavior of the travelers.
\clearpage

\section{Evolutionary Game Theory and Travel Mode Choice}
Evolutionary game theory is used as a vehicle for discussing travel mode choice based on the following apparent similarities: 
\paragraph{}A group can be a substitute for an individual as a participant in evolutionary game theory, and the proportions of the individuals choosing different pure strategies in the group can substitute for mixed strategy. The results of travel mode choice are group behavior within the travel mode subsystems, and the only proportions of individuals choosing each travel mode are meaningful for management and study.
\paragraph{}Nash equilibrium means that the frequency of the adopted strategies makes the strategy payoffs exactly equal with no one desiring a change in strategy, then the percentage of individuals choosing each different strategy remains stable and reaches equilibrium. In the stable travel context, a travel mode choice will tend to be stable, the Nash equilibrium of the evolutionary game will be changed by the means of traffic control, the construction, and the structure of the transportation system.

\subsection{Evolutionary Game Model}

Inspired by the nested binary logit model used to define mode choice and Zhu et al (2018) study, the multi-agent based mode choice game is represented in extensive form in Figure \ref{fig:555}.
Players in the travel mode choice game are be divided into two main categories: car owners and noncar owners. First, every player chooses whether they own a car or not; then, the car owners will select from one of four modes: car, taxi, bus, or rail, and the noncar owners will only select from taxi, bus, or rail. Furthermore, two sub-games are apparent, a car owners sub-game and non car owners sub-game.

\paragraph{} As mentioned in Chapter 1, the extensive form is defined by the set of players, the strategy sets , and the payoff function.
\begin{itemize}
\item the set of players $N = {1,...,n}$, all travelers are players.
\item The strategy sets of the players are $S_1 = {Car owneer, Noncar owner}$ and $ S_2 = $(Travel by car, Travel by taxi, Travel by bus, Travel by rail).
\item The payoff functions of the players are $f_{car} = u_1$, $f_{taxi} = u_2$, $f_{bus} = u_3$, and $f_{rail} = u_4$
\end{itemize}
\paragraph{}Players use mixed strategy, because it is impossible for them to travel using the same pure strategy mode multiple times with certainty.
\paragraph{}As explained in Chapter 1, the mixed strategy happens when an individual plays one of the pure strategies of a game with a continuous probability $p$ between 0 and 1. As a result, the payoff the of the individual using mixed strategy depends on the probabilities of the mixed strategy.
\paragraph{}Figure 3.1 shows the game model of travel mode choice, we note that in stage 1 of the game $p_c$ and $p_n$ are the probabilities of car owners and non car owners respectively. In stage 2, $r^c_{c}$, $r^{c}_{t}$, $r^c_{b}$, and $r^c_{r}$ are the respective probabilities of car owner traveling by car, taxi, bus or rail. The probabilities of the noncar owner traveling by taxi, bus or rail are $r^n_{t}$, $r^n_{b}$, and $r^n_{r}$.
\paragraph{}The payoff function of the players are the following: 
\begin{equation}
f_{car} = T_{car} C_{car}
\end{equation}
\begin{equation}
f_{taxi} = T_{taxi} C_{taxi}
\end{equation}
\begin{equation}
f_{bus} = T_{bus} C_{bus}
\end{equation}
\begin{equation}
f_{rail} = T_{rail} C_{rail}
\end{equation}
\paragraph{}where travel time averages for car, taxi, bus, and rail are $T_{car}$, $T_{taxi}$, $T_{bus}$, $T_{rail}$, and their average travel costs are $C_{car}$, $C_{taxi}$, $C_{bus}$, $C_{rail}$ respectively.
\begin{figure}
 
  \centering
  \begin{tikzpicture}[baseline] % baseline makes the example number stay at the top of the tree
   \Tree[.N [.\textit{Car Owner} [.Car \textit{$U_1$} ][.Taxi \textit{$U_2$} ][.Bus \textit{$U_3$} ][.Rail \textit{$U_4$} ]][.\textit{Non Car Owner} [.Taxi \textit{$U_2$} ][.Bus \textit{$U_3$} ][.Rail \textit{$U_4$} ]]]
     \end{tikzpicture}%
  \caption{Travel mode choice game\label{fig:555}}
\end{figure}
\subsection{Nash Equilibrium of Travel Mode Choice Game}
According to the Folk Theorem mentioned in Chapter 1, any payoff vector satisfying individual rationality can be obtained through a set of specific subgame perfect equilibriums in an infinitely repeated game. In Figure \ref{fig:555} there are two subgame: Car owner subgame and noncar owner subgame, as shown in figures \ref{fig:3} and \ref{fig:4}.The Nash equilibrium of the game is a subgame perfect Nash equilibrium of each sub-game. As mentioned in Chapter 1, backward induction is the method for solving extensive form games and obtaining the Nash equilibrium.

\subsubsection{Nash Equilibrium of Car Owner Subgame}The key feature of mixed strategy Nash equilibrium is that the expectations of the pure strategies are equal, that is, in car owner subgame of figure \ref{fig:3}. \\

\begin{figure}[!h]
  \centering
  \begin{tikzpicture}[baseline] % baseline makes the example number stay at the top of the tree
   \Tree[.\textit{Car Owner} [.Car \textit{$U_1$} ][.Taxi \textit{$U_2$} ][.Bus \textit{$U_3$} ][.Rail \textit{$U_4$} ]]
     \end{tikzpicture}%
  \caption{Car owner subgame\label{fig:3}}
\end{figure}
The products of the travel mode's payoffs and its probabilities are equal and the sum of their probabilities is 1:
\begin{equation}\label{eq:8}
\mu_1 r^c_{c} = \mu_2 r^{c}_{t} = \mu_3 r^c_{b} = \mu_4 r^c_{r}
\end{equation}
\begin{equation}\label{eq:9}
\mu_1 r^c_{c} +  \mu_2 r^{c}_{t} + \mu_3 r^c_{b} + \mu_4 r^c_{r} = 1
\end{equation}

Solving \ref{eq:8} and \ref{eq:9}
\begin{equation}\label{eq:5555}
r^c_{c} = \frac{1}{1+ (\mu_1 / \mu_2)+(\mu_1 / \mu_3)+(\mu_1 / \mu_4)}
\end{equation}
\begin{equation}
r^c_{t} = \frac{1}{1+ (\mu_2 / \mu_1)+(\mu_2 / \mu_3)+(\mu_2 / \mu_4)}
\end{equation}
\begin{equation}
r^c_{b} = \frac{1}{1+ (\mu_3 / \mu_1)+(\mu_3 / \mu_2)+(\mu_3 / \mu_4)}
\end{equation}
\begin{equation}
r^c_{r} = \frac{1}{1+ (\mu_4 / \mu_1)+(\mu_4 / \mu_2)+(\mu_4 / \mu_3)}
\end{equation}

\subsubsection{Nash Equilibrium for Noncar Owners Subgame}
Using backward induction properties as explained in Chapter 1 to solve the subgame shown in Figure \ref{fig:4}. \\

\begin{figure}[!h]
  \centering
  \begin{tikzpicture}[baseline] % baseline makes the example number stay at the top of the tree
   \Tree[.\textit{Non Car Owner} [.Taxi \textit{$U_2$} ][.Bus \textit{$U_3$} ][.Rail \textit{$U_4$} ]]
     \end{tikzpicture}%
  \caption{Non car owner subgame\label{fig:4}}
\end{figure}
The products of the travel mode's payoffs and its probabilities are equal, and the sum of their probabilities is 1, resulting in: 
\begin{equation}\label{eq:10}
\mu_2 r^n_{t} = \mu_3 r^n_{b} = \mu_4 r^n_{r}
\end{equation}
\begin{equation}\label{11}
\mu_2 r^n_{t} + \mu_3 r^n_{b} + \mu_4 r^n_{r} = 1
\end{equation}
Solving \ref{eq:10} and \ref{11} 
\begin{equation}
r^n_{t} = \frac{1}{1+(\mu_2 / \mu_3)+(\mu_2 / \mu_4)}
\end{equation}
\begin{equation}
r^n_{b} = \frac{1}{1+ (\mu_3 / \mu_2)+(\mu_3 / \mu_4)}
\end{equation}
\begin{equation}\label{eq:666}
r^n_{r} = \frac{1}{1+ (\mu_4 / \mu_2)+(\mu_4 / \mu_3)}
\end{equation}
\subsubsection{Nash Equilibrium of Travel Mode Choice Game}
The payoffs of the car owner and the non car owner are their overall expectations. Using backward induction, the products of the payoffs and probabilities are equal and the sum of their probabilities is one: 
\begin{equation}\label{eq:12}
r_n(\mu_2 r^n_{t} + \mu_3 r^n_{b} + \mu_4 r^n_{r}) =  r_c(r^c_{c} + r^{c}_{t} + r^c_{b} + r^c_{r})
\end{equation}
\begin{equation}\label{eq:13}
p_c + p_n = 1
\end{equation}
Solving \ref{eq:12} and \ref{eq:13}
\begin{equation}
p_c = \frac{\mu_2 r^n_{t} + \mu_3 r^n_{b} + \mu_4 r^n_{r}}{\mu_2 r^n_{t} + \mu_3 r^n_{b} + \mu_4 r^n_{r} + \mu_1 r^c_{c} +  \mu_2 r^{c}_{t} + \mu_3 r^c_{b} + \mu_4 r^c_{r}}
\end{equation}
\begin{equation}
p_n = \frac{\mu_1 r^c_{c} +  \mu_2 r^{c}_{t} + \mu_3 r^c_{b} + \mu_4 r^c_{r}}{\mu_1 r^c_{c} +  \mu_2 r^{c}_{t} + \mu_3 r^c_{b} + \mu_4 r^c_{r} + \mu_2 r^n_{t} + \mu_3 r^n_{b} + \mu_4 r^n_{r}}
\end{equation}
\paragraph{}The proportion of travel by car for the traveler is the product of its probability and the probability of car owners traveling by car, and the same goes through other modes:  
\begin{equation}\label{eq:14}
\gamma_{car} = p_c r^c_c
\end{equation}
\begin{equation}
\gamma_{taxi} = p_c r^c_t + p_n r^n_t
\end{equation}
\begin{equation}
\gamma_{bus} = p_c r^c_b + p_n r^n_b
\end{equation}
\begin{equation}\label{eq:15}
\gamma_{rail} = p_c r^c_r + p_n r^n_r
\end{equation}
Substituting equations \ref{eq:5555} to \ref{eq:14}
\begin{equation}
p_{c} = \frac{\frac{3}{(1/\mu_2)+(1/\mu_3)+(1/\mu_4)}}{\frac{4}{(1/\mu_1)+(1/\mu_2)+(1/\mu_3)+(1/\mu_4)}+\frac{3}{(1/\mu_2)+(1/\mu_3)+(1/\mu_4)}}
\end{equation}
\begin{equation}
p_n{} = \frac{\frac{4}{(1/\mu_1)+(1/\mu_2)+(1/\mu_3)+(1/\mu_4)}}{\frac{4}{(1/\mu_1)+(1/\mu_2)+(1/\mu_3)+(1/\mu_4)}+\frac{3}{(1/\mu_2)+(1/\mu_3)+(1/\mu_4)}}
\end{equation}
\paragraph{}The equations above represent the Nash Equilibrium of the travel mode choice game. Looking through equations \ref{eq:14} to \ref{eq:15} we can see that there is a relationship between the individual's payoff and their proportion. That is, as its payoff is increasing, the proportion is decreasing. However, in the last four equations there is a relationship between the proportion and the payoffs of all the travel modes.
The essence of evolutionary analysis is to discuss how the probability changes when one side of the game changes. The learning ability of travelers, which is usually reflected by the tendency dynamic characteristics, in order to determine the change rate.
\section{Constructing the model}
This model is an agent based model of artificial agents playing the travel mode choice game. Each agent occupies a single place in the game and could interact with other neighbor agents. Agents in this game will update their state based on their game choices. An agent's state can be either a car owner or a non car owner which makes him a public transport user. If a non car owner switches their state to a car owner, they have to go through "purchasing a car" according to the probability function. This probability to buy a car function is adopted from an agent based computational approach (Epstein, 2002).
\section{Model Analysis}
\subsection{Travel Cost}
The cost of traveling is an important factor in TMC, although, travel time can be an affecting factor. Measures like public transport fares, car utility, fuel and parking costs have been implemented to estimate travel cost.
\subsection{Travel Time}
Travel time is the time consumed when a traveler moves between two places in a network and is applicable in all transport modes. Two main differences exist in travel time. 
For bus, taxi, and rail travel time is divided into walking time, waiting time, in vehicle time, and transfer time. For car owners, travel time is transfer time. 
\subsection{Payoff}
Travelers optimize between the time and cost of travel. The payoff of travel can be analyzed through the product of average travel cost and travel time in each mode, as shown in equations 3.1 to 3.4.

\section{Experiments and results}
For this experiment we are using Netlog\footnote{Netlog is a multi-agent programmable modeling environment}, the artificial environment is adequate for modeling complex systems which evolve over time. We are going to run the simulation during 200 tick.

In this experiment, we want to test all components available for modifying the passenger's behavior. As we want to remain the characteristics of each mode.
Using Behavior Space tool in Netlogo that allows to make multiple simulations. It runs the model several times, with the ability to change the parameters of the model and record the results after each run.
\begin{table}[h!]
\centering
\begin{tabular}{lllll}
\hline
\multicolumn{1}{l}{Travel mode} & \multicolumn{1}{l}{Average travel time} & \multicolumn{1}{l}{Average travel cost} & \multicolumn{1}{l}{Payoff} & Nash~ Equilibrium  \\ 
\hline
Car                             & 29.3                                    & 10                                      & 293                        & 0.23               \\
Taxi                            & 50                                      & 25                                      & 1250                       & 0.10               \\
Bus                             & 46                                      & 3                                       & 138                        & 0.31               \\
Rail                            & 59.8                                    & 4.5                           & 269.1                      
& 0.36                   
\end{tabular}
\caption{Nash Equilibrium of TCM}
\label{table:2}
\end{table}
To find the Nash equilibrium, we can get it by averaging the travel time and cost then use the Nash equilibrium equations from 3.20 to 3.23 and calculate the payoff in equations from 3.1 to 3.4. The results are shown in figure \ref{table:2}.

When the structure of travel mode choice reaches Nash equilibrium values in Table \ref{table:2}, travel mode will remain stationary unless the payoffs of one or more mode parameters are changed. If a mode changes condition, a small perturbation appears. Then Nash equilibrium will be reached again after self adjusting.


It is important to note that traffic and weather conditions affect the payoff of travel by increasing the time of travel. Travelers tend to choose rail mode because of its relatively stable travel time in weather conditions.
%\subsection{Model Validation}
%An important step in developing agent based models is to evaluate and validate the model carefully. Most models have several types of behaviors going at the same time, and it is important to understand what drives what and what affects what. Although our model is simple, it still needs to be validated and evaluated.




\chapter*{Conclusion}
The goal of this project was to build a travel mode choice model capable of embedding game theory, the simulation model can be used in further research as a starting point. The model assumes that the travelers use mixed strategy when making choices, hence, the probability of a mode being chosen is dynamic. \\

\paragraph{}
The simulations used in this work modeling a set of travelers in making choices of two modes : car or public transport. This model is based on a split mode framework often used to describe the choice behavior of travelers.
\paragraph{}
Game theory has proven its usefulness in modeling the relationship between travel mode choices and their payoffs through Nash Equilibrium. An increase in a mode's payoff can increase its proportion. The evolution part of this study is the change in Nash Equilibrium of travel mode choice.
\paragraph{}
Although people often use multimodal transport, combining two or more travel modes in the same trip, which this model lacks. This mechanism is often used to avoid traffic or lack of coverage in some areas.
\paragraph{}Finally, borrowing concepts from machine learning and artificial intelligence studies, we can use neural networks and genetic algorithms to represent the decision making process and consequent behaviors of travelers. Instead of fixed rules governing the way an individual makes decisions. It would also be interesting to add more components besides the ones that this travel mode choice model have, which can be done by further research. Studying the dynamics of travel mode choice behavior could also extract new patterns that can be out of the scope of this model.

\begin{thebibliography}{2}
\bibitem{Boyd and Richerson, 1985}
Boyd, R. and Richerson, P. J. (1985). Culture and the Evolutionary Process. University of Chicago Press.
\bibitem{Maynard} 
Maynard Smith, J. (1982). Evolution and the Theory of Games. Cambridge University Press, Cambridge.
\bibitem{Bravo et al, 2009}
Bravo et al, (2009) An integrated behavioral model of the land-use and transport systems with network congestion and location externalities. Transportation Research Part B. pp. 584-596.
\bibitem{smith} 
Maynard Smith, J. and Price, G. R. (1973). The logic of animal conflict. Nature, 246:15–18. 
\bibitem{Quijano et al, 2017}
Quijano, N., Ocampo-Martinez, C., Barreiro-Gomez, J., Obando, G., Pantoja, A., and Mojica-Nava, E. (2017). The role of population games and evolutionary dynamics in distributed control systems: The advantages of evolutionary game theory. IEEE Control Systems Magazine, 37(1):70–97.
\bibitem{Litman,2011}
Litman, T. (2011). Why and how to reduce the amount of land paved for roads
and parking facilities. Environmental Practice. 13(1), pp. 38-46.
\bibitem{Marden and Shamma, 2015} 
Marden, J. R. and Shamma, J. S. (2015). Game theory and distributed control. In Young, H.P. and Zamir, S., editors, Handbook of Game Theory with Economic Applications, volume 4, chapter 16, pages 861–899. Elsevier, Amsterdam.
\bibitem{Nelson and Winter, 1982}
Nelson, R. R. and Winter, S. G. (1982). An Evolutionary Theory of Economic Change. Harvard University Press.

\bibitem{Richard, David, 1982}
Richard, B. David, M. and Richard, W(1982). A Selective Review of Travel-Mode Choice Models, pp. 370-375.
\bibitem{Jackson, Leyton-Brown and Shoham; 2013}
Jackson, Leyton-Brown and Shoham, 2013 . Game Theory, https://www.coursera.org/learn/game-theory-1/, accessed on May 2020. 

\bibitem{Von Neumann, and Morgenstern, 1944}
Von Neumann, J. Morgenstern ,O, (1944). Theory of games and economic behavior. Princeton University Press.

\bibitem{Chaoqun Wu, Yulong Pei and Jingpeng Gao , 2015}
Chaoqun Wu, Yulong Pei, Jingpeng Gao, "Evolution Game Model of Travel Mode Choice in Metropolitan", Discrete Dynamics in Nature and Society, vol. 2015, Article ID 638972, 11 pages, 2015. 


\bibitem{Zhu Bai, M. Huang, S. Bian, and Huandong Wu}
Zhu Bai, M. Huang, S. Bian, and Huandong Wu, (2018). A study of taxi service mode choice based on evolutionary game theory. School of Transportation Engineering, Shenyang Jianzhu University.
\bibitem{Joshua M.Epstein}
Epstein, J. (2002). Modeling Civil Violence: An Agent-Based Computational Approach. Proceedings of the National Academy of Sciences of the United States of America, 99(10), 7243-7250. Retrieved October 10, 2020, from http://www.jstor.org/stable/3057848
\end{thebibliography}

\end{document}
\documentclass[10pt,a4paper]{report}
\usepackage[utf8]{inputenc}
\usepackage{listings}
\usepackage{amsmath}
\usepackage{amsfonts}
\usepackage{amssymb}
\begin{document}
\chapter{State of the Art}
\section{Game Theory}
Game Theory is the study of rational behavior in situations involving interdependence as it may involve:
\begin{itemize}
\item Common interest (coordination);
\item Competing interests (rivalry);
\item Rational behavior: players can do the best they can, in their own eyes;
\item Because of the players' interdependence, a rational decision in a game must be based on a prediction of others' responses;
\end{itemize}
\subsection{The parts of a Game} 
A Game consists of three parts : 
Players, Actions and Payoffs
\subsubsection{Players}
Players are the decision makers and they can be : People, Governments or Companies.
\subsubsection{Actions}
What can the players do ?
Decide when to sell a stock, decide how to vote or enter a bid in an auction...
\subsubsection{Payoffs}
Payoffs can represent the motivation of the players, for example : Do they care about profit ? or Do they care about other players ? 
\subsection{Defining Games} Games can be represented in two methods : Normal forms and Extensive Forms.
\subsection{Extensive Form}
An extensive form game includes timing of moves. 
Players move sequentially, represented as a tree.
\begin{itemize}
\item Chess: white player moves, then black player can see white's move and react...
\end{itemize}
Keeps track of what each player knows when he or she makes a decision :
\begin{itemize}
\item Poker: bet sequentially - what can a given player see when they bet. 
\end{itemize}
\subsection{Normal Form: Matrix form or strategic form}
A normal form represents a list of what players get on function of their actions.
Finite, n-person normal form game  ⟨$N, A, u$⟩:
\begin{itemize}
\item Players: $ N = {1, ... , n} $ is a finite set of $n$, indexed by $i$.
\item Actions set for player $i$ $A_i$
\subitem $a = (a_1,...,a_n) \in A = A_1 * ... * A_n $ is an action profile.
\item Utility function or Payoff function for player $i: u_i : A $  $\to$ ${\Bbb{R}}$
\subitem $u = (u_1,..., u_n)$, is a profile of utility functions.
\end{itemize}
\subsection{Best Response and Nash Equilibrium}
\paragraph{Best Response (Definition):}
$a_i^* \in BR(a_{-i}) $ iff $  \forall a_i \in A_i, u_i(a_i^*,a_{-i}) \geq u_i(a_i, a_{-i})$  
\paragraph{Nash Equilibrium (Definition):}
$a = <a_1,...,a_n>$ is a "\textbf{pure strategy}" if $\forall i, a_i \in BR(a_{-i})$
\subsection{Dominant strategies}
let $s_i$ and $s_i^`$ be two strategies for player $i$, and let $S_{-i}$ be the set of all possible strategy profiles for other players.
\bigbreak
\title{\textbf{Definitions:} }
\begin{itemize}
\item $s_i$ \textbf{strictly dominates} $s_i^`$ if $\forall s_{-i} \in S_{-i}, u_i(s_i, s_{-i}) \>> u_i(s_i^`, s_{-i})$
\item $s_i$ \textbf{very weakly dominates} $s_i^`$ if $\forall s_{-i} \in S_{-i}, u_i(s_i, s_{-i}) \geq u_i(s_i^`, s_{-i})$
\item A strategy is called \textbf{dominant} if it dominates all others.
\item A strategy profile consisting of dominant strategies for every player must be a Nash Equilibrium.
\end{itemize}
\subsection{Pareto Optimality}
\title{\textbf{Definition:}}
An outcome $o^*$ is \textbf{Pareto-optimal} if there is no other outcome that Pareto-dominates it.
\section{Mixed Strategies and Nash Equilibrium}
\title{\textbf{Definition:}}
A strategy $s_i$ for agent $i$ as any probability distribution over the actions $A_i$.
\begin{itemize}
\item \textbf{pure strategy:} only one action is played with positive probability
\item \textbf{mixed strategy:} more than one action is played with positive probability
\bigbreak
these actions are called the support of the mixed strategy.
\item Let the set of all strategies for $i$ be $S_i$
\item let the set of all strategy profiles be $S = S_1 \times... \times S_n$
\end{itemize}
\subsection{Utility in Mixed Strategies}
In order to find the payoff if all the players follow mixed strategy profile $s \in S$ we can use the \textbf{expected utility} from decision theory: 
\begin{equation} u_i(s) = \sum_{a \in A}u_i(a)P(a|s)\end{equation}
\begin{equation} P(a|s) = \prod_{j \in N}s_j(a_j)\end{equation}
\subsection{Best Response and Nash Equilibrium}The definitions of best response and Nash equilibrium are generalized from actions to strategies. 

\paragraph{Definition (Best Response): }
$$ s_i^* \in BR(s_{-i}) \textbf{ if }   \forall s_i \in S_i, u_i(s_i^*,s_{-i}) \geq u_i(s_i, s_{-i})$$ 

\paragraph{Definition (Nash Equilibrium)}
$$s = <s_1,...,s_n>\textbf{ is  a }\textbf{Nash Equilibrium if } \forall i, s_i \in BR(s_{-i})$$

\paragraph{Theorem (Nash, 1950)} Every finite game has a Nash equilibrium.
\subsection{Computing Nash Equilibrium}
\paragraph{Two algorithms for finding NE }
\begin{itemize}
\item LCP(Linear Complimentary) [Lemke-Howson].
\item Support Enumeration Method [Porter et al].
\end{itemize}
\subsection{Complexity Analysis}
\title{\textbf{Theorem:}}
Computing a Nash Equilibrium is a \textbf{PPAD-complete}\footnote{PPAD : Polynomial Parity Argument on Directed Graphs} , this theorem has been proven for:
\begin{itemize}
\item for games $\geq$ 4 players;
\item for games with 3 players;
\item for games with 2 players;
\end{itemize}
\subsection{Summary of mixed strategies}
\begin{itemize}
\item Some games have mixed strategy Nash Equilibria.
\item A player must be indifferent between the actions he or she randomizes over.
\item Randomization happen in business interactions, society, sports...
\end{itemize}

\subsection{Strictly Dominated Strategies}
\paragraph{Definition}a strategy $a_i \in A_i $ is strictly dominated by $a'_i \in A_i$ if
\begin{equation} u_i(a_i, a_{-i}) < u_i(a'_i, a_{-i}) ,  \forall   a_{-i} \in A_{-i} \end{equation}
\subsection{Weakly Dominated Strategies}
\paragraph{Definition}a strategy $a_i \in A_i $ is weakly dominated by $a'_i \in A_i$ if
\begin{equation} u_i(a_i, a_{-i}) \leq u_i(a'_i, a_{-i}) ,  \forall   a_{-i} \in A_{-i} \end{equation}
and \begin{equation} u_i(a_i, a_{-i}) < u_i(a'_i, a_{-i}) ,\exists   a_{-i} \in A_{-i} \end{equation} 
\end{document}